%----------------------------------------------------------------------------------------
%	PACKAGES AND DOCUMENT CONFIGURATIONS
%----------------------------------------------------------------------------------------

\documentclass{article}
\usepackage{indentfirst}

\usepackage{graphicx} % Required for the inclusion of images
\usepackage{subfigure} % Required for the inclusion of images
\usepackage{natbib} % Required to change bibliography style to APA
\usepackage{amsmath} % Required for some math elements 
\usepackage{marvosym}
\usepackage{color,amsmath,amssymb,graphicx,fancyhdr,amsfonts,amsthm,algorithmic,verbatim,bbold}
\usepackage{tikz}
\usetikzlibrary{graphs, positioning, quotes, shapes.geometric}
\usepackage{listings}
\lstset{
	language=bash,
	basicstyle=\ttfamily,
	numbers=left,
	numbersep=5pt,
	xleftmargin=20pt,
	frame=tb,
	framexleftmargin=20pt,
	keywordstyle=\color{blue}\bfseries,
	%commentstyle=\color{dkgreen},
	commentstyle=\color{gray!80}\textit,
	stringstyle=\color{red!100!green!50!blue!100},
	breaklines=true,
	%breakatwhitespace=true,
	escapeinside=``,%逃逸字符(1左面的键),用于显示中文例如在代码中`中文...`
	tabsize=4,
	extendedchars=false,
	aboveskip=3mm,
	belowskip=3mm,
	showstringspaces=false,
	columns=flexible,
}

%\usepackage{times} % Uncomment to use the Times New Roman font

%----------------------------------------------------------------------------------------
%	DOCUMENT INFORMATION
%----------------------------------------------------------------------------------------

\title{\textbf{Project 2:  Understanding Cache Memories}} % Title

\author{Zhuohao Li~\textsuperscript{\Letter }\thanks{edith$\_$lzh@sjtu.edu.cn | 519021911248}}% Author name and email

\date{\today} % Date for the report

\begin{document}

\maketitle % Insert the title, author and date

\section{Introduction}

[In this section you should briefly introduce the task in your own words, and what you’ve done in this project. A simple copy from project1.pdf is not permitted.] \\


\section{Experiments}

[This is the main part of your report. It includes three parts and in each part, you need to write concretely, logically but not in full details.]


\subsection{Part A}

\subsubsection{Analysis}

[In this part, you should give an overall analysis for the task, like difficult point, core technique and so on.]

\subsubsection{Code}

[In this part, you should place your code and make it readable in Microsoft Word, please. Writing necessary comments for codes is a good habit.]

\subsubsection{Evaluation}

[In this part, you should place the figures of experiments for your codes, prove the correctness and validate the performance with your own words for each figure’s explanation.]

\subsection{Part B}

\subsubsection{Analysis}

[In this part, you should give an overall analysis for the task, like difficult point, core technique and so on.]

\subsubsection{Code}

[In this part, you should place your code and make it readable in Latex, please. Writing necessary comments for codes is a good habit.]

\subsubsection{Evaluation}

[In this part, you should place the figures of experiments for your codes, prove the correctness and validate the performance with your own words for each figure’s explanation.]

\section{Conclusion}

\subsection{Problems}

[In this part you can list the obstacles you met during the project, and better add how you overcome them if you have made it.]

\subsection{Achievements}

[In this part you can list the strength of your project solution, like the performance improvement, coding readability, partner cooperation and so on. You can also write what you have learned if you like.]



%----------------------------------------------------------------------------------------


\end{document}